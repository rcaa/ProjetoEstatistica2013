\section{An\'alise dos resultados}
\label{sec:resultados}

% Passa-se ao tratamento dos dados por intermédio dos testes estatísticos, os quais dependem das hipóteses a serem testadas. Nesse tópico, é exigido que sejam aplicados testes de hipóteses paramétricos e/ou não paramétricos. Testes de duas amostras são exigidos quando comparando abordagens

Nessa seção, nós mostramos como tratamos os dados da amostra por intermédio dos testes estatísticos. 

Para interpretar os dados, nós primeiramente conduzimos uma análise descritiva
para observar tendência nos dados baseada em algumas características como
dispersão e mediana. Nós elaboramos o gráfico Boxplot ilustrado na
Figura~\ref{fig:boxplot} para comparar a execução das técnicas TE e TG. Esse
gráfico mostra que o tempo de execução dos testes tende a ser menor para TE. A
média para completar as atividades usando TG foi de 975 segundos, enquanto que a
média para TG foi de 824 segundos. A TE teve uma diminuição média de 15\% no
tempo de execução. No mais, nenhuma discrepância ou ponto aberrante apareceu no
gráfico.

\begin{figure}[t]
    \centering
    \includegraphics[width=0.4\textwidth]{images/boxplot.png}
    \caption{}
    \label{fig:boxplot}
\end{figure}

% mostrar o dotplot também aqui
Além do Boxplot, nós também elaboramos o gráfico Dotplot ilustrado na
Figura~\ref{fig:dotplot} para termos uma visão diferente da análise. Com esse
gráfico, podemos ver para cada um dos 18 estudantes, representados de A a R, que
o uso da técnica TE (Específico) para executar os testes levou menos tempo do
que que TG (Genérico). A única exceção foi o estudante P. Outro ponto
importante, é que a funcionalidade (F1 ou F2) usada não influencia nos
resultados.

\begin{figure}[t]
    \centering
    \includegraphics[width=0.4\textwidth]{images/dotplot.png}
    \caption{}
    \label{fig:dotplot}
\end{figure}

Além dos valores da mediana, nós queríamos comparar as observações de acordo com cada resultado das técnicas (TE e TG). Nós pudemos observar para ambas as técnicas que não importa a feature usada para executar os casos de testes, 94\% dos estudantes terminaram em menos tempo usando TE. Dos 18 alunos analisados, só um levou mais tempo para executar os testes com TE.

Continuando com a análise estatística, nós queremos verificar se a tendência
observada nas nossas amostras foi de fato significante. Para verificar isso, nós
rodamos teste de hipótese paramétrico baseado na média. Nesse contexto, nós
usamos o ANOVA como dito na Seção~\ref{sec:metodologia}. Nós executamos o teste
ANOVA e alcançamos um \emph{p-value} para o fator da técnica de
aproximadamente 0,0001, como nosso nível de significância foi de 0,05 esse
resultado nos dá evidência significante de que a técnica TE pode reduzir
o tempo de execução das suítes de testes.

Todo código escrito em R, dados e fontes do texto podem ser acessados livremente no nosso repositório: \\https://github.com/rcaa/ProjetoEstatistica2013