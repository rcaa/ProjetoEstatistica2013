\section{Justificativa}

\label{sec:justificativa}

%De início, explicitam-se os motivos que justificam a pesquisa, determinando-se e delimitando-se o problema, 
%o qual deve estar formulado de maneira clara e precisa


Uma Linha de Produto de Software (LPS) é uma família de software intensivos desenvolvidos a partir de artefatos reusáveis. Por meio do reuso desses artefatos, é possível construir uma grande quantidade de produtos diferentes aplicando composições diferentes de features. Alguns dos benefícios esperados por essa abordagem são: redução de \emph{time-to-market} para lançamento de novos produtos, redução do esforço de manutenção e a melhora da qualidade dos produtos~\cite{pohl-book}.

Nesse contexto, a atividade de testes de LPS tem sido considerada um desafio,
principalmente devido a enorme quantidade de produtos gerados por uma LPS e
também por que os requisitos variam de um produto para o outro de forma que não
existe uma especificação única que contemple todos os possíveis fluxos de execução dos
produtos de uma linha. Para lidar com esses problemas, várias técnicas que
derivam casos de testes funcionais para produtos específicos dentro de uma LPS
tem sido propostas como PLUTO~\cite{Bertolino:pluto} e
ScenTED~\cite{Pohl:scented}. No entanto, apesar de várias propostas de solução, 
essa área de pesquisa ainda não possui uma quantidade satisfatória de avaliações
empíricas que tragam análises e evidências sobre o benefício na prática de se
utilizar casos de teste específicos por produto

Essa falta de discussões pode ser um dos fatores que acabe por desencorajar a
indústria a adotar tais técnicas~\cite{Tevanlinna:spltesting, Engstrom2011}.
Como resultado, pelo que observamos em um ambiente de execução de testes na
indústria, empresas comumente descrevem o comportamento dos produtos de uma
mesma LPS de uma maneira genérica, descrevendo os passos mais comuns entre os
produtos e abstraindo o fato de que alguns passos são opcionais ou alternativos
e algumas vezes até omitindo tais passos. Por exemplo, um caso de teste que
especifica um cenário de uma funcionalidade de geração de relatórios pode conter
todos os possíveis formatos de arquivos de relatório como PDF, HTML, XLS, e
testadores utilizariam esse teste para testar todos os produtos da linha, mesmo
no caso em que os produtos gerados não possuam todas essas opções de formato de
relatórios. 

No entanto, as abstrações contidas nas suítes genéricas podem comprometer a
execução manual dos testes, pois os testadores, que devem seguir estritamente o
passo-a-passo descrito nos testes, podem se confundir. Essa confusão pode levar
a consequências indesejadas tais como defeitos escapados -- aqueles defeitos que
só são descobertos após o lançamento do produto no mercado -- e defeitos que são
reportados mas não existem -- o que afeta a produtividade do processo de
execução dos testes.

Alternativamente, com a adoção de uma técnica de especificação de testes para
uma LPS, seria possível derivar diferentes versões de uma mesma suíte de teste
customizada para as diferentes configurações da linha de produto. Dessa forma,
testadores não se confundiriam durante a execução dos testes e os problemas
mencionados poderiam diminuir. Entretanto, organizações não podem decidir
introduzir novas técnicas ou alterar seus métodos usuais baseadas somente em
suposições. Além disso, não podemos simplesmente assumir que os testes
específicos só trarão benefícios.


Nesse trabalho, temos como objetivo planejar, executar e analisar estudos
comparativos entre as duas técnicas de desenvolvimento de casos de teste. Uma
técnica genérica (TG) observada em um ambiente real de execução de testes e uma
técnica de casos de teste específicos (TE) por produto cujos testes poderiam ser
obtidos por meio de qualquer uma das técnicas de derivação de testes existente.
Para atingir esse objetivo, conduzimos um experimento controlado avaliando o
impacto das técnicas sob o ponto de vista do processo de execução de testes,
coletando métricas relacionadas ao esforço de execução de testes. Após a
análise dos dados coletados, alcançamos resultados que sugerem benefícios, tais
como redução do tempo de execução, que a indústria poderia alcançar ao adotar
uma técnica de derivação de testes (TE).



