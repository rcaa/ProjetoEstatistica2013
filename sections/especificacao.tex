\section{Especifica\c{c}\~ao da amostra}
\label{sec:especificacao}

% Deve-se determinar a área de execução da pesquisa, a população a ser investigada, o tipo de amostra e a determinação do seu tamanho, bem como o tipo de amostragem a ser utilizado. Define-se as variáveis envolvidas.

A área de execução da pesquisa é inserida no contexto de estratégias de testes para Linhas de Produto de Software (LPS)~\cite{pohl-book}. Para atingir os objetivos mencionados na Seção~\ref{sec:objetivos}, um experimento foi definido para investigar como uma população lidaria com as duas estratégias de testes para LPS e consequentemente, trazendo evidências de qual estratégia é a melhor.

Esse experimento foi definido usando a abordagem Goal (objetivo), Question (pergunta) e Metric (métrica) (GQM)~\cite{gqm}. Primeiramente, definimos o objetivo, que é analisar o processo de execução de teste das duas estratégias considerando a eficiência em termos de tempo e o número de \emph{Change Requests (CRs)} terminados. Segundo, definimos as perguntas de pesquisa: (i) A ST reduz o esforço de execução do teste comparado com a GT?, (ii) A ST reduz o número de CRs terminado em comparação com a GT? Por último, definimos uma métrica: número de CRs terminadas.

Para realizar esse experimento definido, utilizamos uma população composta por 18 alunos de graduação e pós-graduação. Eles foram divididos em duplas e usaram as estratégias ST e GT para testar um conjunto de features em uma LPS. Durante esse experimento, coletamos o tempo de execução da suites de testes de cada dupla, cada estudante de uma dada dupla, associados a feature e técnica que usaram para formar nossa amostra. A Tabela~\ref{tab:amostra} exibe uma parte pequena dessa amostra que coletamos e analisamos, como detalhado na Seção~\ref{sec:analise}. As variáveis envolvidas são número da dupla, estudante, feature, técnica e tempo de execução.

\begin{table}[h]\footnotesize
    \caption{Dados da amostra}
    \centering
    \begin{tabular}{|l|l|l|l|l|}
    \addlinespace
    \hline
    {\bf Nº da dupla}  & {\bf Estudante} & {\bf Feature} & {\bf Técnica} & {\bf Tempo de execução}\\ \hline
    1 & 1 & F1 & GT & 880\\ \hline
    1 & 1 & F2 & ST & 841\\ \hline
    \end{tabular}
  \label{tab:amostra}
\end{table}

Nós determinamos um número de 18 pessoas para constituir a população com o intuito de simular como seria um ambiente de testes em uma organização de tamanho médio. Dessa forma, nossa amostra considera 36 entradas, ou seja, 36 linhas diferentes na tabela acima. Ela considera as combinações possíveis entre nossas variáveis.

