\section{Especifica\c{c}\~ao da amostra}
\label{sec:especificacao}

% Deve-se determinar a área de execução da pesquisa, a população a ser investigada, o tipo de amostra e a determinação do seu tamanho, bem como o tipo de amostragem a ser utilizado. Define-se as variáveis envolvidas.

Nesta seção, nós determinamos a área de execução da pesquisa, a população investigada, o tamanho da amostra e as variáveis envolvidas.

A área de execução da pesquisa é inserida no contexto de técnicas de testes para Linhas de Produto de Software (LPS)~\cite{pohl-book}. Para atingir os objetivos mencionados na Seção~\ref{sec:objetivos}, um experimento foi definido para investigar como uma população lidaria com as duas técnicas de testes para LPS e consequentemente, trazendo evidências de qual técnica é a melhor.

Esse experimento foi definido usando a abordagem \emph{Goal} (objetivo),
\emph{Question} (pergunta) e \emph{Metric} (métrica) (GQM)~\cite{gqm}.
Primeiramente, definimos o objetivo, que é analisar o processo de execução de
teste das duas técnicas considerando a eficiência em termos de tempo de
execução. Segundo, definimos a pergunta de pesquisa: A TE reduz o esforço de
execução do teste comparado com a TG? Por último, definimos uma métrica: tempo
de execução dos testes.

Para realizar esse experimento definido, utilizamos uma população composta por
18 alunos de pós-graduação. Definimos dois fatores para bloquear. O primeiro
fator é o aluno já que variações de experiência e motivação podem
influenciar o resultado. O segundo fator é o número de variações existentes
nos casos de tese já que, como explicado anteriormente, quantos mais variações
nos casos de teste mais impacto haverá na técnica TG, o que pode favorecer a
técnica TE. Esse fator é representado pela escolha de duas funcionalidade da
linha para fazer a suíte de testes.

Como tínhamos dois fatores para controlar, escolhemos o \emph{design}
do quadrado latino~\cite{citeulike:2905018} para executar o experimento. Os alunos
foram divididos em duplas e usaram as técnicas TE e TG para testar um conjunto de funcionalidades em uma
LPS. Durante esse experimento, coletamos o tempo de execução da suítes de
testes de cada dupla, cada estudante de uma dada dupla, associados a
funcionalidade e técnica que usaram para formar nossa amostra. A
Tabela~\ref{tab:amostra} exibe uma parte pequena dessa amostra que coletamos e analisamos, como
detalhado na Seção~\ref{sec:analise}. As variáveis envolvidas são número da
dupla, estudante, feature, técnica e tempo de execução.

%\footnote{Os dados completos podem ser obtidos em: https://github.com/rcaa/ProjetoEstatistica2013/tree/master/dados}

\begin{table}[h]\tiny
    \caption{Dados da amostra}
    \centering
    \begin{tabular}{|l|l|l|l|l|}
    \addlinespace
    \hline
    {\bf Nº da dupla}  & {\bf Estudante} & {\bf Funcionalidade} & {\bf Técnica}
    & {\bf Tempo de execução}\\ \hline 1 & 1 & F1 & TG & 880\\ \hline
    1 & 1 & F2 & TE & 841\\ \hline
    \end{tabular}
  \label{tab:amostra}
\end{table}

Nós determinamos um número de 18 pessoas para constituir a população com o
intuito de simular como seria um ambiente de testes em uma organização de
tamanho médio. Dessa forma, nossa amostra considera 36 entradas, ou seja, 36
linhas diferentes na tabela acima. Ela leva em conta as combinações possíveis
entre nossas variáveis.

