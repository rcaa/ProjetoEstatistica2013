\section{Fundamenta\c{c}\~ao te\'orica}

\label{sec:fundamentacao}
%Descreve-se o relacionamento do problema com a teoria que será utilizada na pesquisa
Para entender melhor as duas técnicas apresentadas na seção anterior iremos
descrever nesta seção um exemplo prático de como casos de teste podem ser
genéricos (descrevendo de forma imprecisa o comportamente geral da linha) ou
específicos (descrevendo passos específicos e valores de dado apropriados de
cada produto). Além de mostrar a diferença através do exemplo, também discutimos
quais são as consequências de se utilizar testes genéricos em um ambiente de
execução de testes. 

Para ilustrar as técnicas, utilizaremos um \emph{toy example} de uma LPS que
gera aplcativos que gerenciam conteúdos de multimídia (vídeo, fotos, músicas) em
dispositivos móveis. Através desse aplicativo é possível compartilhar mensagens
com contéudo multimídia anexados, os chamados \emph{Multimedia Messaging
Service} (MMS). Além disso, esses aplicativos ainda podem ser gerados de acordo
com alguns requisitos feitos for uma operadora de telefonia móvel que é cliente.
Vamos chamar essa operadora fictícia de Operadora Azul (OA). O exemplo que
ilustraremos aqui, apesar de não ser real, representa problemas semelhantes aos que foram
observados na prática.

Nosso exemplo considera o cenário de um usuário que envia uma mensagem MMS com
uma foto anexada conforme se detalha na Tabela~\ref{tab:example1}. Esse cenário
de aplica a maioria dos produtos da LPS em discussão. Entretanto, ele não se
aplica aos produtos configurados com os requisitos feitos pela OA. Esse cliente
pede que, antes que uma mensagem MMS seja enviada, uma mensagem apareça na tela
perguntando se o usuário quer realmente enviar aquela mensagem, já que a
tranferência de dados será cobrada. A Tabela~\ref{tab:specexample1} descreve o
comportamento real dos produtos que seguem os requisitos feitos pela OA. Note a
diferença no passo 7 das duas tabelas.


% Table generated by Excel2LaTeX from sheet 'Plan1'
\begin{table}[!t]
  \centering
  \caption{Caso de teste genérico: usuário envia MMS com foto anexada.}
    \begin{tabular}{|c|p{5cm}|p{5cm}|}
    \hline
    \textbf{Passo} & \textbf{Ação do usuário} & \textbf{Resposta do sistema}
    \\
    \hline
    1    & Vá ao menu principal  & Menu principal aparece  \\
    \hline
    2    & Vá ao menu de mensagens  & Menu de mensagens aparece  \\
    \hline
    3    & Selecione ``Criar nova mensagem''  & Tela de edição de mensagens abre
    \\
    \hline
    4    & Adicione destinatário  & Destinatário é adicionado
    \\
    \hline
    5    & Selecione ``Inserir foto''  & Menu de inserir fotos abre
    \\
    \hline
    6    & Selecione uma foto  & A foto é selecionada \\
    \hline
    7    & Seleciona ``Enviar mensagem''  & A mensagem é corretamente enviada
    \\
    \hline
    \end{tabular}
  \label{tab:example1}
\end{table}

% Table generated by Excel2LaTeX from sheet 'Plan1'
\begin{table} [!t]
  \centering
  \caption{Caso de teste específico para produtos configurados com a
  funcionalidade da OA.}
    \begin{tabular}{|c|p{5cm}|p{5cm}|}
    \hline
    \textbf{Passo} & \textbf{Ação do usuário} & \textbf{Resposta do sistema}
    \\
    \hline
    1    & Vá ao menu principal  & O menu principal aparece  \\
    \hline
    2    & Vá ao menu de mensagens  & Menu de mensagens aparece  \\
    \hline
    3    & Selecione ``Criar nova mensagem''  & Tela de edição de mensagens abre  \\
    \hline
    4    & Adicione destinatário  & Destinatário é adicionado \\
    \hline
    5    & Selecione ``Inserir foto''  & Menu de inserir fotos abre \\
    \hline
    6    & Selecione uma foto  & A foto é selecionada \\
    \hline
    7    & Seleciona ``Enviar mensagem''  & Tela de diálogo aparece: 
    ``Tem certeza que deseja enviar essa mensagem? A transferência de dados
    será cobrada'' \\
    \hline
    8    & Pressione ``Sim''  &  A mensagem é corretamente enviada
    \\
    \hline
    \end{tabular}
  \label{tab:specexample1}
\end{table}

Na TG o caso de teste da Tabela~\ref{tab:specexample1} serveria na prática para
testar todos os produtos da linha e o testador observaria uma saída inesperada
do sistema enquanto testando produtos configurados seguindo os requisitos da OA.
Ao contrário, utilizando a TE, existiram os dois testes aqui descritos. O
primeiro, detalhado pela Tabela~\ref{tab:example1}, serviria para testar os
produtos não configurados com a funcionalidade imposta pela OA e, o segundo,
descrito pela Tabela~\ref{tab:specexample1}, serviria para testar aqueles
produtos configurados com a funcionalidade da OA.

Em testes manuais no estilo caixa-preta, quando a especificação não descreve o
que acontece na prática com o sistema, isso provavelmente significa que o caso
de teste revelou uma falha do sistema. No entanto, quando testamos um produto da
OA, não é isso que acontece utilizando o teste genérico descrito. Nesse caso o
produto funciona corretamente para os propósitos da OA. O problema é que o teste
de caso genérico não considera todos os passos necessários para a conclusão do
cenário para esse produto específico. A implementação está correta, mas a
especificação é vaga, de forma que representa de forma imprecisa diferentes
produtos da linha. Nesse contexto, quando um testador não está familiarizado com
as especificidades dos produtos antes da execução do teste (cenário comum
devido a rapidez com que os produtos evoluem), ele pode interpretar a imprecisão
do teste como sendo um defeito do produto. Esse engano pode ser resolvido caso o
testador investigue e encontre evidência que a descrição do teste não se aplica
aquele produto. Contudo, se ele não obtiver essa evidência, ele irá reportar um
defeito no produto que não existe, gastando seu tempo e o tempo de outra pessoa
que deverá analisar o defeito para consertá-lo.

Além de reportar defeitos inválidos, um tipo diferente de problema pode
acontecer. Imaginemos que o produto configurado com a funcionalidade da OA não
apresente a mensagem de alerta no momento da execução do teste. Neste caso o
produto não foi corretamente implementado e a especificação genérica é vaga.
Existe um defeito, mas o testador não vai ser capaz de identificá-lo e
reportá-lo. Se o defeito não for reportado, o produto pode ser lançado no
mercado sem cumprir as exigências feitas pelo cliente. Esse cenário é pior do
que reportar defeitos inválido já que afeta diretamente a qualidade do produto
final, enquanto que reportar defeitos inválidos afeta a produtividade do
processo de execução de testes.

Por fim, pela nossa observação da prática, vimos que a TG impacta o
desenvolvimento de LPS em dois aspectos, qualidade e produtividade. Qualidade
porque alguns defeitos podem escapar para o público e produtividade por conta do
tempo perdido pela equipe de testes devido as investigações e pelos relatórios
de defeitos inválidos. Quanto mais imprecisos forem os testes, mais significante
será o impacto no processo de execução dos testes.
